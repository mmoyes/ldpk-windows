\documentclass[10pt,a4paper]{article}
\usepackage{amsmath}
\usepackage{amsfonts}
\usepackage{amscd}
\usepackage{german}
\usepackage{a4}
\usepackage{bbm}

\pagestyle{plain}
\pagenumbering{arabic}

\newtheorem{theorem}{Theorem}[section]
\newtheorem{lemma}[theorem]{Lemma}
\newtheorem{proposition}[theorem]{Proposition}
\newtheorem{corollary}[theorem]{Corollary}
\newenvironment{proof}[1][Proof]{\begin{trivlist}
    \item[\hskip \labelsep {\bfseries #1}]}{\end{trivlist}}
\newenvironment{definition}[1][Definition]{\begin{trivlist}
    \item[\hskip \labelsep {\bfseries #1}]}{\end{trivlist}}
\newenvironment{example}[1][Example]{\begin{trivlist}
    \item[\hskip \labelsep {\bfseries #1}]}{\end{trivlist}}
\newenvironment{remark}[1][Remark]{\begin{trivlist}
    \item[\hskip \labelsep {\bfseries #1}]}{\end{trivlist}}

\newcommand{\qed}{\nobreak \ifvmode \relax \else
          \ifdim\lastskip<1.5em \hskip-\lastskip
          \hskip1.5em plus0em minus0.5em \fi \nobreak
          \vrule height0.75em width0.5em depth0.25em\fi}

\newcommand{\parity}{\mathrm{par}}

\newcommand{\beq}{\begin{equation}}
\newcommand{\eeq}{\end{equation}}
\newcommand{\bea}{\begin{eqnarray}}
\newcommand{\eea}{\end{eqnarray}}

% Beruehmte Leute
\newcommand{\Euklid}{{\sc{Euklid}}}
\newcommand{\euklid}{{\sc{euklid}}}
\newcommand{\Fourier}{{\sc{Fourier}}}
\newcommand{\Euler}{{\sc{Euler}}}
\newcommand{\Hesse}{{\sc{Hesse}}}
\newcommand{\Lie}{{\sc{Lie}}}
\newcommand{\Newton}{{\sc{Newton}}}

% Reelle Zahlen
\newcommand{\Rgen}{\mathbb{R}}
\newcommand{\Rpow}[1]{\mathbb{R}^{#1}}

\newcommand{\Mat}[2]{\mathrm{Mat}_#1(#2)}
\newcommand{\Bpi}{B_{\pi}}

% Komplexe Zahlen
\newcommand{\Cgen}{\mathbb{C}}

% Quaternionen
\newcommand{\Hgen}{\mathbb{H}}
\newcommand{\Hunit}{\hat{\mathbb{H}}}
\newcommand{\Hunitpos}{\hat{\mathbb{H}}^+}
\newcommand{\Hunitneg}{\hat{\mathbb{H}}^-}
\newcommand{\Haeq}{\hat{\mathbb{H}}^{\mathrm{Eq}}}
\newcommand{\Hnonzero}{\mathbb{H}^{\neq 0}}

% Gruppen und Algebren
\newcommand{\SO}[1]{\mathrm{SO}(#1)}
\newcommand{\SU}[1]{\mathrm{SU}(#1)}
\newcommand{\Con}[1]{\mathrm{Con}(#1)}
\newcommand{\GL}[1]{\mathrm{GL}(#1)}
\newcommand{\so}[1]{\mathrm{so}(#1)}

% Sphaeren
\newcommand{\Sphaere}[1]{\mathrm{S}^{#1}}

\newcommand{\dual}[1]{{}^\ast #1}
\newcommand{\skal}[2]{#1 \cdot #2}

\DeclareMathOperator{\trace}{tr}
\DeclareMathOperator{\symm}{symm}

\newcommand{\parallelprojektor}[1]{\Pi^{\parallel#1}}
\newcommand{\orthogonalprojektor}[1]{\Pi^{\perp#1}}
% Z.B. fuer die "Sphaerischen Tensoren" dritter Stufe
\newcommand{\parallelprojektordrei}[1]{\Pi^{\parallel#1}}
\newcommand{\orthogonalprojektordrei}[1]{\Pi^{\perp#1}}
\newcommand{\orthogonalprojektorvier}[1]{\Pi^{\perp#1}}

\newcommand{\einheitsprojektor}{\hat\Pi}

% Funktor f\"ur Tangentialraeume
\newcommand{\tangential}{\mathrm{T}}

\newcommand{\id}[1]{\mathrm{id}_{#1}}
\newcommand{\derive}[1]{\frac \partial {\partial #1}}
\newcommand{\derivetwo}[2]{\frac \partial {\partial #1}\frac \partial {\partial #2}}
\newcommand{\deriveat}[2]{\left.\frac \partial {\partial #1}\right|_{#2}}
\newcommand{\derivetwoat}[3]{\left.\frac \partial {\partial #1}\frac \partial {\partial #2}\right|_{#3}}
\newcommand{\norm}[1]{\left|#1\right|}

\newcommand{\stern}[1]{{}^\ast{#1}}
\newcommand{\sternvec}[1]{{}^\ast\vec{#1}}

\newcommand{\hatvecq}{\hat{\vec{q}}}
\newcommand{\OmegaQ}{\Omega(q)}
\newcommand{\sinOmegaQ}{\sin_{\Omega}}
\newcommand{\cosOmegaQ}{\cos_{\Omega}}
\newcommand{\sinOmegaQHalb}{\sin_{\Omega/2}}
\newcommand{\cosOmegaQHalb}{\cos_{\Omega/2}}
\newcommand{\sinPowTwoOmegaQHalb}{\sin^2_{\Omega/2}}
\newcommand{\cosPowTwoOmegaQHalb}{\cos^2_{\Omega/2}}
\newcommand{\sinPowThreeOmegaQHalb}{\sin^3_{\Omega/2}}
\newcommand{\cosPowThreeOmegaQHalb}{\cos^3_{\Omega/2}}
\newcommand{\normVecQ}{{\norm{\vec{q\,}}}}
\newcommand{\vecQuad}[1]{{\vec{#1\,}^2}}

\newcommand{\matrixtwo}[4]{\left(\begin{matrix} #1 & #2 \\ #3 & #4 \end{matrix}\right)}


\newcommand{\hlinks}{{h^\mathrm{L}}}
\newcommand{\hrechts}{{h^\mathrm{R}}}

\newcommand{\qrot}[1]{q^{(#1)}}

\newcommand{\source}[1]{{\tt{#1}}}

\newcommand{\vectwo}[2]{\left(\begin{array}{c}#1\\#2\end{array}\right)}

\setcounter{tocdepth}{6}
\setcounter{secnumdepth}{6}

\setlength{\parindent}{0pt}

\title{Proposal for a general lens distortion model}
\author{The Academy of Science-D-Visions}
\date{\today}

\begin{document}

\maketitle

\section{Definitions}
For convenience, we introduce a few convenient abbreviations.
We define the index set $I(a,b)$ for $a,b\in \mathbb(N)$ as:
\beq
I(a,b) = \{i | i \ge a \wedge i \le b\}
\eeq
Similarly, we define an index set with every second index omitted:
\beq
I_2(a,b) = \{i | i \ge a \wedge i \le b\ \wedge \exists j\in\mathbb{N} :  i = 2 j + a\}
\eeq
We define a parity function
\beq
\parity	: \mathbb{Z} \rightarrow \{0,1\} : j \mapsto
		\left\{
			\begin{array}{ll}
			0 & \text{if $j$ is even}\\
			1 & \text{if $j$ is odd}
			\end{array}
		\right.
\eeq

\section{Model}
In this section be construct by definition a mapping which is used to {\bf{remove}} distortion from an image.
The inverse mapping, which is evaluated by an iterative method, is used to {\bf{apply}} distortion to an image.
We start with a general polynomial expression for both components
\bea
x		& \mapsto	& a^{(x)}_{00} \nonumber\\
		& +		& a^{(x)}_{10} x	+ a^{(x)}_{11} y \nonumber\\
		& +		& a^{(x)}_{20} x^2	+ a^{(x)}_{21} xy	+ a^{(x)}_{21} y^2 \nonumber\\
		& +		& a^{(x)}_{30} x^3	+ a^{(x)}_{31} x^2y	+ a^{(x)}_{32} xy^2	+ a^{(x)}_{33} y^3 \nonumber\\
		& +		& a^{(x)}_{40} x^4	+ a^{(x)}_{41} x^3y	+ a^{(x)}_{42} x^2y^2	+ a^{(x)}_{43} xy^3	+ a^{(x)}_{44} y^4 \nonumber\\
		& +		& \ldots \\
y		& \mapsto	& a^{(y)}_{00} \nonumber\\
		& +		& a^{(y)}_{10} x	+ a^{(y)}_{11} y \nonumber\\
		& +		& a^{(y)}_{20} x^2	+ a^{(y)}_{21} xy	+ a^{(y)}_{21} y^2 \nonumber\\
		& +		& a^{(y)}_{30} x^3	+ a^{(y)}_{31} x^2y	+ a^{(y)}_{32} xy^2	+ a^{(y)}_{33} y^3 \nonumber\\
		& +		& a^{(y)}_{40} x^4	+ a^{(y)}_{41} x^3y	+ a^{(y)}_{42} x^2y^2	+ a^{(y)}_{43} xy^3	+ a^{(y)}_{44} y^4 \nonumber\\
		& +		& \ldots
\eea
In the following we will apply some restrictions in order to get a more reasonable form for this deformation.
First of all, for a lens distortion model we postulate that there be a fixed point $(x_c,y_c)$, which we call lens center:
\beq
(x_c,y_c) \mapsto (x_c,y_c)
\eeq
The fundamental theorem of algebra garantees that we can extract the fixed point factor and get the following form (with modified coefficients):
\bea
x	& \mapsto	& x_c + (x - x_c) \sum_{j=0}^N \sum_{k=0}^j a^{(x)}_{jk} (x-x_c)^{j-k} (y-y_c)^k\nonumber\\
y	& \mapsto	& y_c + (y - y_c) \sum_{j=0}^N \sum_{k=0}^j a^{(y)}_{jk} (x-x_c)^{j-k} (y-y_c)^k.
\eea
It would be nice to express both components in one expression. Therefore we set $q=(x,y)$ and define a matrix
\beq
Q_{q_c} = \begin{pmatrix}
	x - x_c	&& 0 \\
	0	&& y - y_c
	\end{pmatrix},
\eeq
so that we can write the deformation as product of a matrix and a tupel ($a_{jk}$ has two components):
\bea
q	& \mapsto	& q_c + Q_{q_c} \sum_{j=0}^N \sum_{k=0}^j a_{jk} (x-x_c)^{j-k} (y-y_c)^k\nonumber\\
\eea
where $q_c = (x_c,y_c)$.
We can encode the fixed point shift in a shift function $h$:
\beq
h(q) = q-q_c
\eeq
which has the inverse
\beq
h^{-1}(q) = q+q_c
\eeq
This simplifies our notation for the lens distortion model:
\beq
q \mapsto h^{-1} \circ g \circ h(q).
\eeq
where
\beq
g(q) = Q_0 \sum_{j=0}^N \sum_{k=0}^j a_{jk} x^{j-k} y^k
\eeq
This is the most compact form of our raw lens distortion model.
Having extracted the fixed point by means of $h$ we will only consider
the simplified function $g$ in the following.
The raw form has several disadvantages. First, the coefficients
do not reflect the symmetry of the lens (if there is any).
Second, it is unintuitive, since we cannot guess a priori, which
of the coefficients are important and which are likely to be less important.
Our intention is now to express the deformation function by coefficients which allow a classification by symmetry.
As we will see later this can be achieved by polar coordinates:
\begin{align}
x	& = r \cos \phi = r c_{\phi}\nonumber\\
y	& = r \sin \phi = r s_{\phi}
\end{align}
where
\begin{align}
r &= \sqrt{x^2+y^2} \nonumber\\
\phi &= \arctan(y,x)
\end{align}
In polar coordinates, the mapping $g$ reads:
\bea
g(q) = Q_0 \sum_{j=0}^N  r^j \sum_{k=0}^j a_{jk} s_\phi^{j-k} c_\phi^k \nonumber\\
\eea
We will transform the inner sum by means of a complex \Fourier-expansion.
We will show that the sum
\beq
\sum_{k=0}^j a_{jk} s_\phi^{j-k} c_\phi^k
\eeq
has the following \Fourier-series:
\beq
\sum_{k\in I_2(-j,j)} d_{j,k} e^{ik\phi}
\eeq
where for all $k$:
\beq
d_{j,k} = d_{j,-k}^\dagger.
\eeq
The purpose of this lemma is to draw the connection between parity of the monomial $r^j$ and the angular frequency
that appears in the deformation function for that power;
Even powers of $r$ only occur with even frequencies, odd powers of $r$ occur with odd frequencies.
\begin{proof}
elsewhere
\end{proof}
The deformation now reads:
\bea
g(q) = Q_0 \sum_{j=0}^N  r^j \sum_{k\in I_2(-j,j)} d_{j,k} e^{ik\phi}
\eea
Obviously, this double sum is ordered with respect to powers of $r$. We will reorder it now with respect
to angular frequencies. To do so, we need to get the index sets correctly:
\bea
\sum_{j=0}^N\hspace{0.5cm} \sum_{k\in I_2(-j,j)} \cdots
	& = & \sum_	{
			\begin{array}{c}
			j,k \\
			0\le |k| \le j \le N \\
			\parity{(j)}=\parity{(k)} \\
			\end{array}
			} \cdots \nonumber\\
	& = & \sum_	{
			\begin{array}{c}
				k \\
				0\le |k| \le N
			\end{array}
			}\hspace{0.5cm}
			\sum_	{
				\begin{array}{c}
				j \\
				|k|\le j \le N \\
				\parity{(j)}=\parity{(k)} \\
				\end{array}
				}
			\cdots \nonumber\\
	& = & \sum_{k=-N}^{N}\hspace{0.5cm} \sum_{j\in I_2(|k|,N)} \cdots \nonumber\\
\eea
We rewrite the deformation accordingly:
\beq
g(q) = Q_0 \sum_{k=-N}^{N} e^{ik\phi} \sum_{j\in I_2(|k|,N)}  r^j  d_{j,k}
\eeq
For the inner sum we shift the summation index by $|k|$ and extract $r^{|k|}$:
\bea
g(q) = Q_0 \sum_{k=-N}^{N} e^{ik\phi} r^{|k|} \sum_{j\in I_2(0,N-|k|)} r^j  d_{j+|k|,k}
\eea
Finally, we split the sum over $k$ into $\cos$ and $\sin$ terms in order to get a manifestly real-valued distortion model.
\begin{align}
g(q)	& =	& Q_0 \Big\{	&   \sum_{j\in I_2(0,N)}  r^j  a_{0j} \nonumber\\
	&	&		& + \sum_{k=1}^{N} \sin(k\phi) r^{k} \sum_{j\in I_2(0,N)}  r^j  b_{kj} \nonumber\\
	&	&		& + \sum_{k=1}^{N} \cos(k\phi) r^{k} \sum_{j\in I_2(0,N)}  r^j  a_{kj} \Big\}
\end{align}
Let us unroll this for the first six angular frequencies (i.e. $N=6$) in order to get a feeling for what is going on here.
We keep in mind that all coefficients $a$ and $b$ are 2-tupels like $p$.
\begin{align}
g(q)	& = Q_0 \Big[		& 		& a_{00} + a_{02} r^2 + a_{04} r^4 + a_{06} r^6 \nonumber\\
	&			& \sin(\phi) r	& \big(	b_{10} + b_{12} r^2 + b_{14} r^4 \big) \nonumber\\
	&			& \cos(\phi) r	& \big( a_{10} + a_{12} r^2 + a_{14} r^4 \big) \nonumber\\
	&			& \sin(2\phi) r^2	& \big(b_{20} + b_{22} r^2 + b_{24} r^4 \big) \nonumber\\
	&			& \cos(2\phi) r^2	& \big(a_{20} + a_{22} r^2 + a_{24} r^4 \big) \nonumber\\
	&			& \sin(3\phi) r^3	& \big(b_{30} + b_{32} r^2 \big) \nonumber\\
	&			& \cos(3\phi) r^3	& \big(a_{30} + a_{32} r^2 \big) \nonumber\\
	&			& \sin(4\phi) r^4	& \big(b_{40} + b_{42} r^2 \big) \nonumber\\
	&			& \cos(4\phi) r^4	& \big(a_{40} + a_{42} r^2 \big) \nonumber\\
	&			& \sin(5\phi) r^5	& \big(b_{50} \big) \nonumber\\
	&			& \cos(5\phi) r^5	& \big(a_{50} \big) \nonumber\\
	&			& \sin(6\phi) r^6	& \big(b_{60} \big) \nonumber\\
	&			& \cos(6\phi) r^6	& \big(a_{60} \big) \Big]
\end{align}
In the uppermost line you see the purely radial symmetric part - there is no angular dependency.
For a perfectly arranged radially symmetric lens system, only these coefficients make sense.
The lines containing even angular frequencies (2,4,6) represent symmetric deformations around
the horizontal axis and the vertical axis of the image. These terms can be present in perfect anamorphic cameras.
The odd angular frequencies (1,3,5) can appear in unperfectly arranged lens systems, e.g. when the anamorphic prism
is inclined or displaced in some way.

\section{Inverse function}

\section{Relation to 3DEV3's lens distortion model}
\end{document}


